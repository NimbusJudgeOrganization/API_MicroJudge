\section{Horas em segundos e minutos}
\arquivoProblema{lst1exerC\_nome\_matricula}

Construa um programa em C que leia um número inteiro de horas e mostre ao 
usuário a quantos minutos e quantos segundos estas horas correspondem.

\subsection*{Entrada}

A primeira linha terá um número inteiro \textit{hora}.

Os casos de testes irão variar os valores de \textit{hora} da seguinte forma:

$$0 \leq \textit{hora} \leq 2^{32}-1$$

\subsection*{Saída}

Seu programa deve exibir a \textit{hora} em minuto e a \textit{hora} em segundos,
cada um separado por uma quebra de linha.

Observe os casos de exemplos para melhor entendimento da saída.

\subsection*{Exemplos}

\exemplo{tests/in51.test}{tests/out51.test}

\exemplo{tests/in52.test}{tests/out52.test}

\exemplo{tests/in6.test}{tests/out6.test}


% vim: set nocindent formatoptions+=aw tw=72:
